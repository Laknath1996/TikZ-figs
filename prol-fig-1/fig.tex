\documentclass{article}

\usepackage{caption}
\usepackage{subcaption}
\usepackage{graphicx}
\usepackage{tikz}
\usetikzlibrary {arrows.meta,automata,positioning}

\begin{document}
\def\seqone{{1,1,1,1,1,1,1,0,1,0}}
\def\seqtwo{{1, 0, 1, 0, 1, 0, 1, 1, 1, 1}}

\begin{figure}
    \begin{subfigure}[b]{0.475\textwidth}
        \centering
        \resizebox{\linewidth}{!}{
            \begin{tikzpicture}
                \draw (1, 2) node[circle,draw=green!90,fill=green!20, very thick, minimum size=1, label=Bernoulli$(p)$] {1};
                \tikz \foreach \i in {0,...,9}
                    \draw (\i+1,0) node[circle,draw=green!90,fill=green!20, very thick, minimum size=1] {\pgfmathparse{\seqone[\i]}\pgfmathresult};
                \draw (0.5, 0.3) node {$\dots$};
            \end{tikzpicture}
        }
        \caption{Scenario 1}   
    \end{subfigure}
    \hfill
    \begin{subfigure}[b]{0.475\textwidth}
        \centering
        \resizebox{\linewidth}{!}{
            \begin{tikzpicture}
                \draw (1, 2) node[circle,draw=green!90,fill=green!20, very thick, minimum size=1, label=Bernoulli$(p)$] {1};
                \draw (4, 2) node[circle,draw=red!90,fill=red!20, very thick, minimum size=1, label=Bernoulli$(1-p)$] {1};
                \tikz \foreach \i in {0,...,9}{
                    \pgfmathsetmacro{\col}{ifthenelse(Mod(\i,2)==1,"red","green")};
                    \draw (\i+1,0) node[circle,draw=\col!90,fill=\col!20, very thick, minimum size=1] {\pgfmathparse{\seqtwo[\i]}\pgfmathresult};
                };
                \draw (0.5, 0.3) node {$\dots$};
            \end{tikzpicture}
        }
        \caption{Scenario 2}   
    \end{subfigure}
    \par\bigskip\bigskip\bigskip\bigskip\bigskip
    \begin{subfigure}[b]{0.475\textwidth}
        \centering
        \resizebox{\linewidth}{!}{
            \begin{tikzpicture}[statenode/.style={circle, draw=blue!60, fill=blue!5, very thick, minimum size=1}]
                \draw (1, 2) node[statenode] (state0) {0};
                \node[statenode] (state1) [right=of state0] {1};
                \path[->] (state0) edge [bend left, above] node{$0.9$} (state1);
                \path[->] (state1) edge [bend left, below] node{$0.9$} (state0);
                \path[->] (state0) edge [loop above] node{$0.1$} (state0);
                \path[->] (state1) edge [loop above] node{$0.1$} (state1);
                \tikz \foreach \i in {0,...,9}
                \draw (\i+1,0) node[circle,draw=blue!90,fill=blue!20, very thick, minimum size=1] {\pgfmathparse{\seqone[\i]}\pgfmathresult};
                \draw (0.5, 0.3) node {$\dots$};
            \end{tikzpicture}
        }
        \caption{Scenario 3}   
    \end{subfigure}
    \hfill
    \begin{subfigure}[b]{0.475\textwidth}
        \centering
        \resizebox{\linewidth}{!}{
            \begin{tikzpicture}
                \tikz \foreach \i in {0,...,9}
                    \draw (\i+1,0) node[circle,draw=orange!90,fill=orange!20, very thick, minimum size=1] {\pgfmathparse{\seqone[\i]}\pgfmathresult};
                \draw (0.5, 0.3) node {$\dots$};
            \end{tikzpicture}
        }
        \caption{Scenario 4}   
    \end{subfigure}
\end{figure}
\end{document}